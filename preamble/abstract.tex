\chapter*{Abstract}
\Huge
\begin{centering} 
\textbf{}
\end{centering}
\normalsize\\
% This is a template for a Harvard engineering thesis.\\For errors, corrections or clarifications please contact \href{mailto:smeijer11@gmail.com}{smeijer11@gmail.com}.\\
% No prior knowledge of \LaTeX{} is expected, just take a look around and try to work out how it works! There is lots of documentation in both the source code of this file and the text so I would suggest you look at both while reading! \\(You can get to the source code of this abstract by navigating to preamble/abstract.tex [on Overleaf this is in the file explorer on the left of the page or sometimes you can double click on a word in the document on the right to take you to the place]) %% Hello! this is the sourcecode!
% I would suggest you now go and look in main.tex now to understand how the document is laid out.
% \\Good luck with your work, I hope this is useful to you!

Understanding  spatial  openness  in  residential  housing  is  important  for  enhancing  living  quality  and  guiding architectural design. 
This study proposes a multifaceted evaluation method of spatial openness in rental housing by leveraging large-scale data from Tokyo’s 23 wards. 
Visibility Graph Analysis (VGA) is applied to floor plan images  to  assess  two-dimensional  spatial  connectivity.  
In  addition,  semantic  segmentation  is  performed  on interior photos to capture three-dimensional openness features such as window ratios and ceiling visibility. 
The study investigates how openness characteristics vary by geographic location and construction year, identifying spatial  and  temporal  patterns.  
Finally,  it  explores  the  relationship  between  these  openness  features  and subjective housing preferences, providing insights into how spatial openness influences perceived residential quality.


